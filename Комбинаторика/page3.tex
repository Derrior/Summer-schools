\centerline{\bf Комбинаторика 3}
\begin{enumerate}
\item Рассмотрим квадратную табличку. Изначально черепашка стоит в левом верхнем углу, с него и
будем отсчитывать координаты.
\\ $(a, b)$ - клетка на пересечении $a$-й сверху строки и $b$-го слева
столбца.
\begin{enumerate}
\item Запишем в клетку $(i, j)$ количество способов для черепашки добраться до этой клетки. Чему оно
равно? Обозначим его $[i, j]$
\item Заполним несколько диагоналей и покрутим листок в руках, ничего не напоминает?
\item найдите значения сумм $ \sum_{i = 0}^k [i, k - i],\quad \sum_{i = 0}^k (-1)^i \cdot [i, k - i]$
\end{enumerate}
\item Вам сказали купить 7 булок в магазине. Вы увидели, что в
ассортименте есть три вида булок, причём каждого вида по 10 штук. Сколько различных наборов вы можете
принести домой? А если вы решили ещё и купить хотя бы по одной булке каждого вида?
\item Сколькими способами можно составить колоду из $n$ карточек, где на каждой карте написано число от
1 до $m$? (Это обобщение предыдущей задачи)

\definement{Правильной скобочной последовательностью} длины $2n$ (сокращённо {\bf ПСП})
называется последовательность из открывающих {\tt'('} и закрывающих {\tt ')'} скобок, такая, что:
\begin{itemize}
\item В последователности количество открывающих скобок равно количеству закрывающих
\item На каждом префиксе открывающих скобок не меньше, чем закрывающих.
\end{itemize}
Это формальное определение соотносится с интуитивным - каждая открытая скобка будет закрыта, и
каждая закрытая была открыта.

\definement{n-м числом Каталана} является количество ПСП длины $2n$. Обозначается
$C_n$.
\item Сколько ПСП длины 2? длины 6? длины 8?
\item Выведите рекурсивную формулу для чисел Каталана. {\small (Подсказка: надо отцепить некоторый
префикс)}
\item Выведите формулу для чисел Каталана через числа сочетаний (Которые $C_n^k$)
\begin{enumerate}
\item Научитесь изображать скобочные последовательности длины $2n$, c $n$ открывающими скобками, но
не обязательно правильные, как пути из $(0, 0)$ в $(n, n)$ на плоскости.
\item Поймите, что отличает правильные последовательности от неправильных
\item Примените силу симметрии и решите основную задачу.
\end{enumerate}
\end{enumerate}
