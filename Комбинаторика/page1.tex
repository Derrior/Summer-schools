\centerline{\bf Комбинаторика 1}
\begin{enumerate}
\item На кухне 3 различных чашки, 2 разных блюдца и 7 разных чайных ложек. Сколькими способами можно
составить набор из чашки и блюдца? Из чашки и ложки? Из чашки, блюдца и ложки?
\item Сколько двузначных чисел состоят только из чётных цифр (не обязательно различных)?
Трёхзначных? $N$-значных?
\item Сколько двузначных чисел можно составить из нечётных цифр, использую каждую цифру не более
одного раза? Трёхзначных? Трёхзначных, при условии, что цифры идут в порядке возрастания? 7-значных?

\definement{Перестановкой} длины $n$ называется последовательность из $n$
различных чисел от 1 до $n$.
\item Сколько перестановок длины 4? длины 6? длины k?
\item Сколько различных слов, не обязательно осмысленных, можно составить, переставляя буквы слова
    {\bf ПРИМУС}? {\bf СТРАУС}? \\{\bf ПЕРЕПЕЛ}? {\small (подсказка: посчитайте ответ для очень
        коротких слов)}
\item Сколькими способами можно посадить 7 человек вокруг круглого стола? (способы $a$ и $b$ считаются
    одинаковыми, если $a$ можно получить, вращая $b$ вокруг стола)
\item Сколькими способами можно раскрасить грани куба в 6 цветов, чтобы не было двух граней одного
    цвета? (способы $a$ и $b$ считаются одинаковыми, если можно так изменить положение куба, покрашенного
        способом $a$, в пространстве, что получится куб, покрашенный в способ $b$) Тот же вопрос для
        октаэдра и 8 цветов.

        Октаэдр - правильный восьмигранник, представляющий из себя
        две склеенные по основанию пирамидки, у которых основанием служил квадрат (как в Египте).

\definement{Префиксом} строки или последовательности называется некоторое начало
        её. Например, у строки {\tt КАРП} есть следующие префиксы: {\tt '', 'К', 'КА', 'КАР','КАРП'}
\end{enumerate}
