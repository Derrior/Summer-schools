\centerline{\bf \large Графы 1}
\sloppy
\begin{enumerate}
\definement{Граф} -- абстрактный математический объект, представляющий собой множество {\it «вершин»} и
множество {\it «рёбер»}, каждое из которых соединяет пару вершин. Легко представлять граф на бумаге как
точки и линии, соединяющие их. \underline{Петля} -- ребро, соединяющее вершину саму с собой.
\underline{Кратные рёбра} -- рёбра, соединяющие одну и ту же пару вершин. Граф без петель и кратных рёбер называется
{\it простым}.
\item Филч каждую ночь обходит коридоры Хогвартса так, что ни по какому из коридоров не проходит
дважды. В эту ночь на перекрёстке перед спальней Гриффиндора он побывал 5 раз. Сколько коридоров
ведут к спальне, если Филч:
\begin{enumerate}[noitemsep]
\item не с неё начал и не на ней закончил;
\item с неё начал, но не на ней закончил;
\item с неё начал и на ней закончил?
\end{enumerate}
\item Рон учится играть в шахматы на доске $3 \times 3$. Гермиона поставила в верхние углы доски два белых
коня, а в нижние -- два чёрных и спросила: «Можно ли сделать так, чтобы кони белого цвета стояли на
одной диагонали, а чёрного -- на другой?». Рон ответил на вопрос, а вы?

\definement{Степень} вершины $v$ -- число входящих-выходящих из неё рёбер (петли считают дважды).
\item В честь праздника Дамблдор провесил гирлянды между восемью башнями Хогвартса так, что от
каждой идёт гирлянда ровно к трём другим. Сколько всего на это у него ушло гирлянд?
\item Гарри заметил, что Дамблдор забыл о девятой башне -- Директорской. У Гарри есть запасные
гирлянды, но сможет ли он перевесить гирлянды так, чтобы от каждой башни снова вело ровно три
гирлянды?

\definement{Полный} граф -- граф, в котором между каждой парой вершин есть ребро.
\item Ночью привидения замка Хогвартс обожают пожимать друг другу руки и делают это при первой
возможности, но только один раз в сутки. В эту ночь Плакса Миртл насчитала $n$ привидений под потолком
своего туалета. Сколько рукопожатий ей пришлось пронаблюдать?
\item Докажите, что среди любых 6 Уизли найдутся либо 3 попарно знакомых, либо 3 попарно
незнакомых.
\item Профессор Снегг варит противоядие. Получить его можно из трёх разных искристых зелий
одинакового цвета. Как назло, у профессора в кладовой только 17 не искристых зелий. Но как профессор
он знает, какие получаются зелья при смешивании любых двух из 17. Для каждой пары -- разный
результат, но все получившиеся зелья -- искристые красного, зелёного или жёлтого цвета. К сожалению,
у профессора осталось время только на то, чтобы взять три зелья из 17. Удастся ли Снеггу приготовить
противоядие?

\definement{Ориентированный граф} -- граф, на каждом ребре которого задано направление.
\item У Певерелла было 3 сына. Из его потомков 100 имело по 2 сына, два по одному сыну, остальные
умерли бездетными. Сколько потомков было у Певерелла? (Потомки могут быть только мужского пола).
\item В межшкольном турнире по квиддичу каждая команда сыграла с каждой 1 раз. Ничьих не было.
Можно ли пронумеровать команды так, чтобы первая команда выиграла у второй, вторая -- у третьей, и т.
д.?
\item В лавке Олливандера по кругу расставлены коробочки, в каждой -- волшебные палочки (быть
может, ноль). За один ход мастер Олливандер может взять все палочки из любой коробочки и разложить
их, двигаясь по часовой стрелке, начиная со следующей коробочки, кладя в каждую коробочку по одной
палочке.
\begin{enumerate}[noitemsep]
\item Докажите, что если в каждом следующем ходу палочки берут из той коробочки, в которую попала
последняя палочка на предыдущем ходе, то когда-нибудь повторится начальное размещение палочек.
\item Докажите, что за несколько ходов из любого начального размещения палочек по коробочкам можно
получить любое другое (можно брать из любой коробки).
\end{enumerate}
\end{enumerate}
