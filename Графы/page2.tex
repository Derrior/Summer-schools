\centerline{\bf \large Графы 2}
\begin{enumerate}
\setcounter{enumi}{10}
\definement{Путь} -- это последовательность вершин $v_1, v_2, \ldots, v_n$, таких, что между двумя
соседними вершинами на пути есть ребро (в случае ориентированного графа -- ребро в нужную сторону).
Путь называется {\it вершинно-простым}, если все вершины на пути различны, кроме, возможно, $v_1$ и
$v_n$. Путь называется {\it рёберно-простым}, если все рёбра на пути различны.
Путь называется {\it циклом}, если первая и последняя вершины на пути совпадают. Путь называется
{\it простым циклом}, если он цикл и он рёберно-простой.
\item Все знают, что в Запретный лес ходить нельзя, однако там откуда-то всё равно появляются
тропы. Хагрид насчитал 49 тропинок, ведущих от Хогвартса, одну тропинку, выводящую к пещере Арагога,
и по 4 или 6 тропинок на перекрёстках. Докажите, что Хагрид всегда может попасть по тропам от
Хогвартса к Арагогу.

\definement{Связный граф} -- граф, в котором между любой парой вершин есть путь.
\item Арахниды широко распространены по территории Запретного леса, у каждой – своя ловчая паутина,
которую они не любят покидать. Пауки придумали интересный способ общения: от каждой такой паутины
отходит чётное число переговорных нитей к другим паутинам (не обязательно к различным), через эти
нити сигнал передаётся от паутины к паутине, пока не достигнет места назначения. Сейчас с любой
паутины можно передать сообщение на любую другую. Докажите, что одна оборванная паутинка не помешает
паукам общаться.

\definement{Компонентa связности} графа -- множество вершин, достижимых из вершины $v$. Граф
разбивается на непересекающиеся компоненты связности, в частности, связный граф состоит из ровно
одной компоненты.
\item На первом уроке вождения метлы было $n$ учеников. К концу урока оказалось, что каждый
ученик столкнулся не меньше, чем c $\frac{n - 1}{2}$ другими, причём сталкивались только ученики с одного
факультета. Докажите, что в тот раз мадам Трюк вела занятие только у одного факультета.
\item В Выручай-комнате есть $n$ перекрёстков, а сколько там коридоров -- неизвестно. Попав в
Выручай-комнату, Драко Малфой насчитал $\frac{(n - 1)(n - 2)}{2}$ различных коридоров. Докажите, что он побывал на
всех перекрёстках.
\item В каминной сети из одного камина можно попасть в другой только тогда, когда они напрямую
соединены связью, либо существует последовательность связанных каминов, ведущих от первого камина ко
второму. Долорес Амбридж пригрозила закрыть два камина, но каких -- не сказала. К счастью, сеть
устроена так, что при закрытии любых двух каминов всё равно из любого камина можно добраться в любой
другой. Какое минимальное количество связей должно быть в такой сети из $2n$ каминов?
\item Днём из любого перекрёстка Хогвартса можно попасть в любой другой меньше, чем за 5 минут.
При этом в школе нет коридоров, по которым нужно идти 5 и больше минут. Ночью же в одном из
коридоров спит дух Пивз, который перебудит весь замок, если кого-нибудь услышит. Докажите, что любой
переход, который можно совершить, не разбудив духа, можно совершить за 15 минут.
\item Про каждую пару из n учеников Хогвартса можно с уверенностью сказать, друзья они или
враги. Более того, после исследований Гермиона смогла с уверенностью заявить, что выполняются два
правила: «Друг моего друга – мой друг» и «Враг моего друга – мой враг» Разъехавшись по домам на
Рождество, ученики заскучали и отправили сов всем своим друзьям. Известно, что Гарри и Драко --
враги.
\begin{enumerate}
\item Какое наименьшее число писем могло быть отправлено?
\item А наибольшее?
\end{enumerate}
\item Управляя ладьёй, Гарри за 64 хода обошёл все поля шахматной доски $8 \times 8$ и вернулся на
исходное поле. Докажите, что число ходов по вертикали не равно числу ходов по горизонтали.

\end{enumerate}
