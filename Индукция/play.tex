Привет, сегодняшняя тема - индукция.

Метод математической индукции применяется для доказательства утверждений в общем виде,
чтобы не проверять каждый частный случай. Например, неравенство, которое часто применяется в
доказательстве разных вещей, называется Неравенство Бернулли, вот оно:
\[ (1 + \alpha)^n \Ge 1 + n \alpha, \quad \alpha \Ge -1\]
Мы можем легко проверить для каких-нибудь фиксированных маленьких $n$ и $\alpha$, но так не хочется
проверять каждый раз это утверждение, которое вроде бы всегда верно.

С другой стороны, непонятно, как его доказывать. С положительными числами ещё можно сказать, что там
скобки раскрываются, как надо, и все слагаемые положительные.
Но для отрицательных $\alpha$ ничего по-прежнему непонятно. В дальнейшем мы не будем писать $\alpha
\Ge -1$, а запомним это.

Заведём последовательность вспомогательных утверждений: $A_i$ - неравенство верно, если вместо $n$
подставить $i$. Чтобы доказать неравенство, осталось лишь показать, что все вспомогательные
утверждения верны. Их, конечно, многовато, чтобы проверять каждое. Проверим первое из них - $n = 1$.
\[ (1 + \alpha)^1 = 1 + \alpha\]
Неравенство выполнено для всех $\alpha$, ничего не скажешь.

Докажем теперь, что из каждого утверждения $A_i$ следует следующее в цепочке - $A_{i + 1}$. В этом и
заключается основная идея доказательств методом мат. индукции - доказательство утверждений через
предыдущие. Ведь если мы докажем это, то все утверждения окажутся верны: $A_1$ мы уже рассмотрели (и
это важная часть), $A_1 \implies A_2$, значит, и $A_2$ верно, $A_2 \implies A_3$, и т. д.

{\bf Важно}, что мы доказываем для непонятного $i$, ведь так мы докажем много ``стрелочек'' в
следующие утверждения, а не одну, как если бы мы зафиксировали $i$.

Итак, доказательство того, что $A_i \implies A_{i + 1}$. Пусть утверждение $A_i$ верно. Тогда верно, что
\[ (1 + \alpha)^i \Ge 1 + i \cdot \alpha\]
Мы хотим доказать следующее {\it(написать справа на доске)}
\[ (1 + \alpha)^{i + 1} \Ge 1 + (i + 1) \alpha\]
Умножим обе части на $(1 + \alpha)$. Это число неотрицательно из-за нашего ограничения на $\alpha$,
вот мы и можем умножать.
\[ (1 + \alpha)^{i + 1} \Ge (1 + i \alpha) \cdot (1 + \alpha)\]
Слева уже то, что мы хотим.
Раскроем скобки в правой части:
\[(1 + i \alpha) \cdot (1 + \alpha) = (1 + i\alpha + \alpha + i \alpha^2) \xrightarrow{\text{Квадрат
всегда неотрицательный, можем убрать и поставить нужный знак}}\Ge 1 + (i + 1) \alpha \]
Что получаем:
\[(1 + \alpha)^{i + 1} \Ge (1 + i \alpha) \cdot (1 + \alpha) \Ge 1 + (i + 1)\alpha\]

Убираем центральную часть и получаем, что $A_{i + 1}$ верно.

Мы доказали не то, что для любого $i$ $A_{i + 1}$ верно, не совсем. Мы доказали, что если $A_i$
верно, то и $A_{i + 1}$ тоже верно. Но раньше всего этого мы доказывали очень простое утверждение,
частный случай - про $n = 1$. Теперь мы можем быть уверены, что неравенство Бернулли верно для
любого натурального $n$. (порисовать стрелочки).

Немного терминов, облегчающих жизнь всем, кто их использует - доказательство по индукции состоит из
двух частей: База индукции - доказательство первого утверждения, из которого потом будет расти
цепочка стрелочек, и Шаг индукции - Доказательство того, что мы имеем право проводить стрелочки.

При доказательстве шага индукции мы предполагаем, что утверждения для меньших $i$ верны - это
называется предположением индукции.

Нарисовать на доске, где база, где шаг, где пользуешься предположением.
