\centerline{\bf \large Индукция 1}
Во всех задачах данного листка присутствует, пусть иногда и не совсем в явном виде, некоторый
пошаговый процесс. В ходе решения задачи необходимо выделить в таком процессе нечто, не меняющееся
на протяжении всего процесса. Доказательство неизменности этого в задачах данного листка
предлагается проводить по следующему алгоритму:
\begin{enumerate}
\item установить, что нечто верно в начале процесса;
\item объяснить, почему наличие этого в некоторый момент влечет за собой наличие его и после очередного
шага пошагового процесса.
\end{enumerate}
Это позволяет утверждать, что нечто сохраняется на протяжении всего процесса.
\begin{enumerate}
\item Придумайте 10 различных натуральных чисел, чтобы каждые два имели общий делитель, больший 1, но
при этом чтобы НОД всех чисел был равен 1.
\item Докажите, что уголок из трех клеток можно разрезать на 1024 одинаковых уголка такой же формы, но
меньшего размера.
\item Докажите, что квадрат можно разрезать на любое количество квадратов, начиная с шести.
\item Докажите, что если после очередной денежной реформы в России будут введены в обращение
монеты достоинством в 5 и 26 копеек, то пользуясь только ими можно будет уплатить без сдачи любую
сумму, начиная с 1 рубля.
\item Докажите, что прямоугольник, площадь которого делится на 8, а каждая из сторон не менее 2,
можно разрезать на фигурки из четырех клеток в форме буквы «Г».
\item На столе стоят тридцать два стакана с водой. Разрешается взять любые два стакана и уравнять
в них количества воды, перелив часть воды из одного стакана в другой. Как с помощью таких операций
добиться того, чтобы во всех стаканах воды стало поровну?
\item В девять одинаковых мензурок налито до краев девять разных жидкостей (в каждую мензурку —
своя особая жидкость), кроме того, имеется одна пустая мензурка. Можно ли составить одинаковые смеси
этих жидкостей в каждой из девяти мензурок, оставив при этом десятую мензурку пустой? Выливать
жидкости не разрешается, из мензурки можно отмерить и отлить любую часть имеющейся там жидкости.
\item Известно, что в туристическом клубе каждый участник знаком не более чем с пятью другими.
Докажите, что можно разделить участников на не более чем шесть групп для похода выходного дня таким
образом, что ни в какую группу не попадут двое знакомых.
\item Торт разрезали прямолинейными разрезами на несколько кусков. Оказалось, что одна сторона у
ножа была грязная. Докажите, что всегда найдется хотя бы один чистый кусок.
\item Маленькому Коле подарили ножницы. Он отрезал от занавески треугольный кусок, после чего
принялся резать его на части: Коля выбирает какой-нибудь из уже имеющихся кусочков занавески, после
чего разрезает его прямолинейным разрезом на две части. Докажите, что сколько бы Коля так не
работал, среди получившихся кусочков занавески всегда можно будет найти треугольник. Пусть после
нескольких таких разрезаний получилось пять треугольных кусков занавески. Может ли спустя какое-то
время их оказаться только три?
\end{enumerate}
