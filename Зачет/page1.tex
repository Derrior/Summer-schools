\centerline{\Large \bf Зачёт}
\begin{enumerate}
\item В графе из каждой вершины выходит по три ребра. Может ли в нём быть 2017 рёбер?
\item Сколько существует таких натуральных $n$, не превосходящих 2017, что сумма $1^n + 2^n + 3^n + 4^n$
оканчивается на 0?
\item У Марка спросили последние 179 цифр числа $777^{179}$, а вы найдите последнюю цифру!
\item У Егора есть 6 книг, а у Антона – 8. Сколькими способами они могут обменять три книги
Егора на четыре книги Антона?
\item Докажите, что среди любых 6 друзей Димы найдутся либо 3 попарно знакомых, либо 3 попарно
незнакомых.
\item На столе стоят $2^n$ стаканов с соком. Денис может уравнять в любых двух стаканах количество
сока, перелив часть сока из одного стакана в другой. Как с помощью таких операций Денису уравнять
количество сока во всех стаканах?
\item Дима решил всю индукцию, а одну задачу оставил вам. Докажите, что
\[ 1 + 2 + \ldots + n = \frac{n \cdot (n+1)}{2}\]
\item Вокруг нового круглого стола в столовой сидят Денис, Дмитрий, Антон, Егор, Марк и Борис.
Сколькими способами можно посадить их вокруг стола? (способы $a$ и $b$ считаются одинаковыми, если $a$
можно получить, вращая $b$ вокруг стола).
\item Докажите, что существует 179 подряд идущих составных чисел.
\item Сколькими способами можно разбить 14 человек на пары?
\end{enumerate}
