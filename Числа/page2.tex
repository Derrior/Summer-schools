\centerline{\bf \large Числа 2}
\begin{enumerate}
\definement{Вычетом}
называется подмножество целых чисел, дающих одинаковые остатки по какому-либо фиксированному
модулю. С ними можно делать те же операции, что и с целыми числами - складывать, умножать, возводить
в степень. То, что это не число, а вычет, которому принадлежит это число, показывает черта над
числом.  При операциях с ними обычно используются т.н. {\it представители} - просто числа, имеющие
нужный остаток. Например:
\[ \overline{5} + \overline{7} \equiv \overline{12} \equiv \overline{2} \equiv \overline{-8} \mod
10\]

С ними связано множество теорем, используемых в современной криптографии и параллельных вычислениях.

-- Это законно вообще?

-- Да, в предыдущем листке в задачах доказывается корректность сложения и умножения.

\definement{Нулевой вычет:} Вычет чисел, которые делятся на значение модуля.

\definement{Обратным вычетом} $a^{-1}$ для $a$ называется такой вычет $b$, что $a \cdot b \equiv 1
\mod n$. Вычет, у которого есть обратный, называется обратимым.
\item Докажите, что у любого ненулевого вычета по простому модулю $p$ есть обратный.
\item Докажите, что у любого вычета может быть не более одного обратного.
\item Решите уравнения:
\[\bullet \; \overline{5}x \eqmod{7} \overline{3} \quad \bullet \; x^2 \eqmod{4} \overline{0} \quad \bullet \;
x^2 \eqmod{24} \overline{1} \]
\item {\it (Теорема Вильсона)} Пусть $p$ - простое, тогда:
\[(p - 1)! \equiv -1 \mod p\]
\item {\it (Малая Теорема Ферма)} Пусть $p$ - простое, $c $ не кратно $p$. Тогда
\[c^{p - 1} \equiv 1 \mod p\]
\item {\it (Китайская Теорема об остатках)} Пусть $m_1, m_2$ - взаимно простые, $r_1, r_2$ - вычеты
по модулям $m_1, m_2$ соотв. Тогда существует и единственно такое $a$, что
\[
\begin{cases}
    a \equiv r_1 \mod m_1\\
    a \equiv r_2 \mod m_2
\end{cases}
\]
a) Докажите, что такого $a$ в промежутке чисел $[0, m_1 \cdot m_2)$ существует не более одного \\
для фиксированного набора остатков $r_1, r_2$;

b) Докажите утверждение теоремы;

c) Обобщите теорему для случая $n$ чисел.
\definement{Функция Эйлера} $\varphi(n)$ - функция от натурального числа, обозначающая количество
чисел от 1 до $n$, взаимно простых с $n$.

\item Посчитайте значения функции Эйлера:
\[\bullet\; \varphi(5) \quad \bullet \; \varphi(16) \quad \bullet \;\varphi(30) \quad \bullet \;
\varphi(p^n), \text{$p$ - простое} \]
\item Докажите, что для взаимно простых $a, b$ выполняется: {\large$\varphi(a) \cdot \varphi(b) =
\varphi(ab)$}
\item {\it (Теорема Эйлера)} Докажите, что если $(a, n) = 1$, то $a^{\varphi(n)} \equiv 1 \mod n$\\
Указание: рассмотрите все вычеты, взаимно простые с $n$.
\end{enumerate}
