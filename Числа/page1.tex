\centerline{\bf \large Числа 1}
В этом листке все числа считаются целыми, если не сказано иного.
\begin{enumerate}
\definement{Простое число} -- число, у которого ровно два натуральных делителя - 1 и оно само.
\item Докажите, что простых чисел бесконечно много.
\item Докажите, что при любых $a$ и $e$ в арифметической прогрессии вида $a, a + e, a + 2e \ldots$
содержится составное число.

\definement{Деление с остатком} числа $a$ на натуральное число $b$ называется нахождение таких чисел $p, r$, что
выполнено \\ $0 \Le r < b, a = p \cdot b + r$. Число $r$ называется остатком. Иногда говорят, что {\tt
$a$ сравнимо с $r$ по модулю $b$}
\item Докажите, что определение выше корректное, то есть для любой пары чисел $a, b$ подходящая пара
$(p, r)$ существует и единственна.
\item Найдите результат деления с остатком
\begin{equation*}
\bullet \; a = 5, b = 3 \quad \bullet a = 1537, b = 26 \quad \bullet a = -1537, b = 26
\end{equation*}
\item Найдите остаток от деления
\begin{enumerate}[noitemsep]
\item $8^{100}$ на 5
\item $3^{4^{5^{6}}}$ на 7
\end{enumerate}
\definement{НОД} - Наибольший общий делитель двух чисел - НОД$(a, b)$ -- наибольшее
натуральное число, на которое делятся оба числа без остатка.

Фразу {\it$a$ и $b$ имеют одинаковые остатки при делении на $c$} для краткости будем писать так: $a
\underset{c}{\equiv} b $, или $a \equiv b \mod c$. Наибольший общий делитель (НОД) будем писать просто как скобки: НОД$(a, b)$ = $(a, b)$.
Это обозначения, используемые далее в задачах, в своих решениях используйте что хотите.
\item Докажите, что если $a \eqmod{n} b, c \eqmod{n} d$, то:

$\bullet \;a + c \eqmod{n} b + d$

$\bullet \;a \cdot c \eqmod{n} b \cdot d$
\item Когда для числа $a$ существует такое число $b$, что $a \cdot b \eqmod{n} 1$?
\begin{enumerate}[noitemsep]
\item Докажите, что $(a, b) = (b, a - b)$, теперь постройте алгоритм поиска НОДа двух чисел.
\item Найдите алгоритм решения в целых числах уравнения $ax + by = (a, b)$
\\(Неизвестные здесь только $x$ и $y$).\\
\end{enumerate}
{\color{DarkGray}Подсказка: запустите алгоритм из предыдущего пункта.}\\
БОНУС: Напишите общий вид любого решения в целых числах такого уравнения.

с) Решите основную задачу.
\item Докажите, что уравнение $ax \eqmod{p} b, a \neq 0, p $ - простое, всегда 
\begin{enumerate}[noitemsep]
\item имеет решение.
\item имеет единственное решение.
\end{enumerate}
\hr
\item Докажите, что существует 1000 подряд идущих составных чисел
\item Докажите, что среди любых 10 последовательных чисел найдется число, взаимно простое с
остальными.
\end{enumerate}

