\documentclass[12pt]{article}
\usepackage{/home/z/mydefines}

\geometry{
    a4paper,
    left=1cm,
    right=1.5cm,
    top=2cm,
    bottom=2.5cm,
}

\usepackage{fancyhdr}
\setlength{\headheight}{15.6pt}
\pagestyle{fancy}

\lhead{Август 2017 \hfill Алые Паруса \hfill 7-8 класс}

\begin{document}
\centerline{\large \bf Письменная работа №1}
\begin{enumerate}
\item Упростите выражение и посчитайте его значение при $a = 9.3, b = 1.4$
\begin{equation*}
\frac{(a + b)^2(a - b)^2(4a^2 - b^2)}{(2a - b)(a^4 - 2a^2b^2 + b^4)}
\end{equation*}
\item Зайцы Игорь и Антон косят трын-траву. Кстати, заяц Игорь, работая один, скосил бы всю траву за
8 часов, а заяц Антон -- за 24 часа. За какое время зайцы скосят трын-траву, если будут работать
сообща?
\item В компании 3 человека -- Алексей, Борис и Виктор. Каждый из них -- рыцарь или лжец. Рыцарь
всегда говорит правду, лжец всегда лжёт. Один из них украл шар. Алексей сказал: <<Борис -- лжец>>.
Борис сказал: <<Виктор украл шар>>. Виктор сказал: <<Алексей -- рыцарь>>. Кто украл шар, если рыцарь
среди них только один?
\item Сколько существует четырёхзначных чисел, у которых все цифры различны?
\item На шахматной доске $8 \times 8$ в один угол поставили коня, а клетку в противоположном углу вырезали.
Может ли конь обойти всё оставшееся поле, побывав в каждой клетке ровно по одному разу?
\item На стороне $AB \; \Delta ABC$ взята точка $P$, такая что $AP : PB = 2 : 1$, а на стороне $AC$
середина -- точка $Q$. Оказалось, что $CP = 2CQ$. Докажите, что $\angle ABC = 90^{\circ}$.
\item На плоскости дано 50 точек, причём не все они лежат на одной прямой. Через каждую пару точек
проведена прямая. Докажите, что среди этих точек можно выбрать точку, через которую проходит не
менее 8 таких прямых.
\item Докажите, что в компании из  любых 6 человек найдутся либо трое попарно знакомых, либо трое
попарно незнакомых.
\item Числа $p, 2p + 1, 4p + 1$ простые. Чему может равняться $p$? Перечислите все варианты и
докажите, что других нет.
\item Имеется квадратная доска со стороной длины $n$. На ней расставлена не более чем $(n - 1)$
шашка. Разрешается менять местами любые два столбца или строки. Докажите, что с помощью таких
операций их можно расставить по одну сторону от главной диагонали (не на ней).
\end{enumerate}
\end{document}

