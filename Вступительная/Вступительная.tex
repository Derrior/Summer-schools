\documentclass[12pt]{article}
\usepackage{/home/z/mydefines}

\geometry{
    a4paper,
    left=1cm,
    right=1.5cm,
    top=2cm,
    bottom=2.5cm,
}

\usepackage{fancyhdr}
\setlength{\headheight}{15.6pt}
\pagestyle{fancy}

\lhead{Август 2017 \hfill Алые Паруса \hfill 7-8 класс}

\begin{document}
\centerline{\large \bf Письменная работа №1}
\begin{enumerate}
\item Упростите выражение и посчитайте его значение при $a = 9.3, b = 1.4$
\begin{equation*}
\frac{(a + b)^2(a - b)^2(4a^2 - b^2)}{(2a - b)(a^4 - 2a^2b^2 + b^4)}
\end{equation*}
\item Доказать, что для любого натурального $n$ верно
\begin{equation*}
    2^{n - 1} \cdot (x^n + y^n) \Ge (x + y)^n
\end{equation*}
\item На стороне $AB \; \Delta ABC$ взята точка $P$ такая, что $AP : PB = 2 : 1$, а на стороне $AC$
середина -- точка $Q$. Оказалось, что $CP = 2CQ$. Докажите, что $\angle ABC = 90^{\circ}$
\item На первом этапе соревнований по бегу состоялись парные забеги, причём каждый бегун бегал с
каждым ровно один раз. В каждом забеге кто-то выигрывал (ничьих не было).
Теперь букмекеры анализируют эти данные, чтобы заработать кучу денег:
считается, что бегун $A$ {\it может обогнать} бегуна $B$, если выполнено хотя бы одно из условий:
\begin{itemize}
    \item В забеге этих двух бегунов победил бегун $A$.
    \item Есть такой бегун $C$, что в забеге бегунов $A$ и $C$ победил $A$, а в забеге бегунов $C$ и
    $B$ победил $C$
\end{itemize}
Докажите, что найдётся бегун, который {\it может обогнать} любого другого бегуна.
\item В производстве горючего используется т.н. {\it октановое число} - это концентрация октана в
данном горючем, выраженная в процентах. Сколько граммов бензина с октановым числом 98 можно получить
из 196 граммов бензина с октановым числом 80?
\item Найдите остаток при делении на 5 числа {\large $7^{\left(7^{(7^{7})}\right)}$}
\item На шахматной доске 8х8 в один угол поставили коня, а клетку в противоположном углу вырезали.
Может ли конь обойти всё оставшееся поле, побывав в каждой клетке ровно по одному разу?
\item Зайцы Игорь и Антон косят трын-траву. Кстати, заяц Игорь, работая один, скосил бы всю траву за
8 часов, а заяц Антон - за 24 часа. За какое время зайцы скосят всю траву, если будут работать
вместе?
\end{enumerate}
\end{document}
